Prices are driven by four state variables: 

\begin{enumerate}
	\item expected dividend growth,
	\item expected inflation,
	\item short-term real interest rate,
	\item price of risk.
\end{enumerate}

\subsection{Dividend growth, inflation, and the stochastic discount factor}

We are considering six shocks denoted by a vector $\varepsilon_{t}\in\R^{6}$: shocks to dividend growth, to inflation, to \emph{expected} dividend growth, to \emph{expected} inflation, to the real risk-free rate, and to the price of risk.

As usual, we denote by $D_t$ the level of aggregate \emph{real} dividned at time $t$ and $d_t = \log D_t$. Its growth rate follows the following process:
\begin{equation}
	\begin{aligned}
			\Delta d_{t+1} & = z_t + \inner{\sigma_d}{\varepsilon_{t+1}},\\
			z_{t+1} & = (1-\phi_z) g + \phi_z z_t + \inner{\sigma_z}{\varepsilon_{t+1}},
	\end{aligned}
\end{equation}
with $\sigma_d,\sigma_z\in\R^6$ being the vectors of loadings on the shocks; and $\phi_z\in[0,1]$ measures the autocorrelation. Each variable is assumed to be stationary and positively autocorrelated (although the realised dividend growth may be (and in fact will be) negatively autocorrelated). Also, $g$ can be interpreted as the unconditional mean of dividend growth.

We are interested in pricing nominal bonds, so we specify a process for inflation, so we denote the price level by $\Pi_t$ and the log by $\pi_t = \log \Pi_t$. Inflation process is 
\begin{equation}
	\begin{aligned}
		\Delta \pi_{t+1} & = q_t + \inner{\sigma_\pi}{\varepsilon_{t+1}}\\
		q_{t+1} & = (1-\phi_q)\bar{q} + \phi_q q_t + \inner{\sigma_q}{\varepsilon_{t+1}}
	\end{aligned}
\end{equation}
where $\sigma_\pi, \sigma_q\in\R^6$ are the vector of the loadings of shocks and $\bar{q}$ is the unconditional mean of inflation. Again, $\phi_q$ is the autocorrelation. In what follows, all quantities are rexpressed in real terms unless it's specifically specified. Hence, multiplying by $\Pi_t$ converts quantitites from real to nominal terms.

Regarding the discount rates, they are determined by the real risk-free rate and by the price of risk. We denote by $r_{t+1}^f$ the continuous compounded risk-free return between times $t$ and $t+1$, and also assumed that is known at time $t$ and follows
\begin{equation}
	r_{t+1}^f = (1-\phi_r)\bar{r}^f + \phi_r r_t^f + \inner{\sigma_r}{\varepsilon_{t+1}},
\end{equation}
where $\sigma_r\in\R^6$ is the vector of the loadings on the shocks, $\bar{r}^f$ is the unconditional mean of $r_t^f$, and $\phi_r$ is the autocorrelation.

The variable that determines the price of risk, and therefore risk premia in this homoskedastic model is denoted by $x_t$ and follow
\begin{equation}
	x_t = (1-\phi_x)\bar{x} + \phi_x x_t  + \inner{\sigma_x}{\varepsilon_{t+1}}
\end{equation}
A key assumption here is that only fundamental dividend risk is priced directly, which means that the price of risk is proportional to the vector $\sigma_d$.

The \emph{Stochastic Discount Factor} is given by:
\begin{equation}
	M_{t+1} = \e{-r_{t+1}^f - \frac{1}{2}\sigma_d^2x_t^2 - x_t\inner{\sigma_d}{\varepsilon_{t+1}}}.
\end{equation}
Asset prices are therefore determined by means of the Euler equation:
\begin{equation}
	\E_t\left[M_{t+1}R_{t+1}\right] = 1,
\end{equation}
where $R_{t+1}$ denotes the real return on a traded asset.

\subsection{Real bonds}

Let $P_{n,t}^r$ denote the price of an $n$-period real bond at time $t$, i.e.,  it denotes the time-$t$ price of an asset with a fixed payoff of one at time $t+n$. Since there are no intermediate payoffs, its return between $t$ and $t+1$ equals
\begin{equation}
	R_{n,t+1}^r = \frac{P_{n-2,t+1}^r}{P_{n,t}^r}.
\end{equation}
Hence, the prices of real bonds can be determined recursively from the Euler equation by 
\begin{equation}
	\E_{t}\left[M_{t+1}P^r_{n-1,t+1}\right] = P_{n,t}^r,
\end{equation}
with $P_{0,t}^r = 1$.

%Hence, we can summarise the model into the following equations:
%\begin{equation}
%	\left\{
%		\begin{aligned}
%			\Delta d_{t+1} & = z_t + \inner{\sigma_d}{\varepsilon_{t+1}},\\
%			z_{t+1} & = (1-\phi_z) g + \phi_z z_t + \inner{\sigma_z}{\varepsilon_{t+1}},\\
%			\Delta \pi_{t+1} & = q_t + \inner{\sigma_\pi}{\varepsilon_{t+1}}\\
%			q_{t+1} & = (1-\phi_q)\bar{q} + \phi_q q_t + \inner{\sigma_q}{\varepsilon_{t+1}}\\
%			r_{t+1}^f &= (1-\phi_r)\bar{r}^f + \phi_r r_t^f + \inner{\sigma_r}{\varepsilon_{t+1}},\\
%			x_t &= (1-\phi_x)\bar{x} + \phi_x x_t  + \inner{\sigma_x}{\varepsilon_{t+1}}\\
%			&\phantom{=} \sigma_i, \ i\in\{d,\pi, q, r, x\}; \\
%			&\phantom{=} \phi_j, j\in\{z,q,r,x\};\\
%			&\phantom{=}  g, \bar{q}, \bar{r}^f, \bar{x}
%		\end{aligned}
%	\right.
%\end{equation}



